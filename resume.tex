%!TEX TS-program = xelatex
\documentclass[a4paper, extended]{comcv}

\usepackage[english]{babel}

\title{Avinal's CV}
\fullname{Avinal}{Kumar}{}
\cvtitle{Software Engineer, Open Sourcerer}
\website{https://avinal.space}{avinal.space}
\email{ripple@avinal.space}
\github{https://github.com/avinal}{GitHub}
\linkedin{https://www.linkedin.com/in/avinal}{LinkedIn}
\currentdate{Feb 2024}

\begin{document}
\section{Experience}
\combosection{Red Hat}{Software Engineer}{June 2022 -- Present}{
  \begin{tightlist}
    \item Working on Pipeline Service project as a part of Red Hat Trusted Application Pipeline
    \item \href{https://github.com/tektoncd/results}{Tekton Results} Maintainer
  \end{tightlist}
}

\combosection{Red Hat}{Developer Tools Intern}{Jan 2022 -- May 2022}{
    \begin{tightlist}
        \item Designed and implemented the \href{https://github.com/MiniTeks}{MiniTeks} (Minimal Tekton Server) along with unit tests, a training project that creates Tekton resources on Kubernetes/OpenShift clusters using the Tekton API. It was developed using Go, Kubernetes, Tekton, and Redis and had 3 parts: MKS Server, MKS CLI, and MKSDashboard.
        \item Contributed to Tekton Results as part of the Pipeline Service team. Tekton Results aims at long-term, efficient results storage of Tekton PipelineRuns and TaskRuns.
    \end{tightlist}
}

\combosection{API7.ai}{Technical Writer, \href{https://github.com/apache/apisix}{Apache APISIX}}{Feb 2022 -- July 2022}{
    \begin{tightlist}
        \item Redesigning developer and user documentation of the Apache APISIX project. Creating Katacoda tutorials for APISIX collaborating with the community and integrating their feedback into the documentation.
    \end{tightlist}
}

\combosection{Google Summer of Code 2021}{Contributor, \href{https://github.com/fossology/fossology}{The FOSSology Project}}{May 2021 -- Aug 2021}{
    \begin{tightlist}
        \item Upgraded the build system from Unix Makefile to a more optimized and flexible CMake generator.
        \item Migrated the CI/CD of the project from Travis CI to GitHub Actions. This migration removes the dependency on a third-party CI and provides better build time and better integration with GitHub.
        \item Refactored and fixed years-old unit and functional testing code written in C/C++ and PHP. Build time was reduced to 5-7 minutes (twice as fast), and CI time was reduced to 20-25 minutes from 1-2 hours.
    \end{tightlist}
}

\combosection{XResearch}{Java Development Intern}{Jan 2021 -- May 2021}{
    \begin{tightlist}
        \item Designed and developed an Inventory and Billing Management App using Spring Boot and
PostgreSQL. Created REST API endpoints according to the functional requirements of the software.
    \end{tightlist}
}

\combosection{Google Season of Docs 2020}{Technical Writer, VideoLAN}{Sep 2020 -- Nov 2020}{
  \begin{tightlist}
    \item Created \href{https://code.videolan.org/docs/vlc-android-user}{VLC for Android User Documentation}
    \item Documented VLC for Android app using Sphinx, reStructuredText, Markdown, and shell scripting.
    \item I achieved the goal of this project, which was to provide well-researched and user-friendly app documentation enriched with supporting screenshots and step-by-step tutorials.
  \end{tightlist}
}

\section{Projects}
\combosection{\href{https://github.com/avinal/Profile-Readme-Wakatime}{Automated WakaTime GitHub Actions App}}{Python, Docker, GitHub Actions, Bash Scripting}{}{
  \begin{tightlist}
    \item Built an automated WakaTime coding statistics update system using Python scripts, shell scripts, Docker, the WakaTime API, and GitHub Actions. Currently, it has 42 stars on GitHub and 200+ active users worldwide. WakaTime is a utility to track coding activities across multiple IDEs and machines.
    \item This app generates a colorful bar graph of the coding activity tracked by WakaTime in the last week, daily at a specified time in SVG format. This image file can then be embedded into READMEs and websites. It can be scheduled to update more often or less often.
  \end{tightlist}
}

\combosection{\href{https://github.com/avinal/xeus-basic}{Xeus-BASIC}}{C, C++, Jupyter Notebook}{}{
  \begin{tightlist}
    \item Built a Jupyter Kernel for the BASIC language using the Xeus Framework, C, and C++. As of now, it can execute BASIC programs line by line in Jupyter Notebook and output their results.
  \end{tightlist}
}

\combosection{\href{https://github.com/avinal/FITS-Image}{Astronomical Image Extraction from FITS File}}{C++, CFITSIO, Boost.GIL, CMake}{}{
  Flexible Image Transport System has been used for decades to store and transfer astronomical images and needs specialized software to view them, this project extract those images as JPEG or PNG that can be viewed by common image viewer.
  }
\vspace{\topsep}

\section{Education}
\combosection{National Institute of Technology, Hamirpur}{B.Tech in Computer Sci. and Eng.}{2018 -- 2022}{CGPA: 8.45/10

  \ifextended
    Courses - Advanced Calculus, Statistics, Probability and Queuing Theory, Algorithm Design, Digital Logic Design, Database Management Systems, Data Structures, Operating Systems, Compiler Design, Computer Networks
  \else
  \fi
  }

\vspace{\topsep}

\section{Skills}
\combosection{Programming Languages}{}{}{
  Go, Elm, C, C++, Rust, Bash Scripting
}
\vspace{\topsep}
\combosection{Technologies}{}{}{
  \begin{tightlist}
    \item GNU/Linux, Fedora, CI/CD, Vim, CMake
    \item Docker, Kubernetes, OpenShift, Tekton CD
  \end{tightlist}
}


\end{document}